
%\begin{appendices}

%report class wants \appendix instead
\appendix
\chapter{Example profiles}

This appendix prints some of the example profiles that ship with the \RAppArmor
package. To load them in \AppArmor, an ascii file with these rules needs to be
copied to the \texttt{/etc/apparmor.d/} directory. After adding new profiles to
this directory they can be loaded in the kernel by running \texttt{sudo service
apparmor restart}. The \texttt{r-cran-rapparmor} package that can be build on
Debian and Ubuntu does this automatically during installation. Once profiles
have been loaded in the kernel, any user can apply them to an \R session using
either the \texttt{aa\_change\_profile} or \texttt{eval.secure} function from
the \texttt{RAppArmor} package.

\section{Profile: \texttt{r-base}}
\label{r-base}

\footnotesize
\begin{verbatim}
#include <tunables/global>
profile r-base {
        #include <abstractions/base>
        #include <abstractions/nameservice>

        /bin/* rix,
        /etc/R/ r,
        /etc/R/* r,
        /etc/fonts/** mr,
        /etc/xml/* r,
        /tmp/** rw,
        /usr/bin/* rix,
        /usr/lib/R/bin/* rix,
        /usr/lib{,32,64}/** mr,
        /usr/lib{,32,64}/R/bin/exec/R rix,
        /usr/local/lib/R/** mr,
        /usr/local/share/** mr,
        /usr/share/** mr,
}
\end{verbatim}


\section{Profile: \texttt{r-compile}}
\label{r-compile}

\footnotesize
\begin{verbatim}
#include <tunables/global>
profile r-compile {
        #include <abstractions/base>
        #include <abstractions/nameservice>

        /bin/* rix,
        /etc/R/ r,
        /etc/R/* r,
        /etc/fonts/** mr,
        /etc/xml/* r,
        /tmp/** rmw,
        /usr/bin/* rix,
        /usr/include/** r,
        /usr/lib/gcc/** rix,    
        /usr/lib/R/bin/* rix,
        /usr/lib{,32,64}/** mr,
        /usr/lib{,32,64}/R/bin/exec/R rix,
        /usr/local/lib/R/** mr,
        /usr/local/share/** mr,
        /usr/share/** mr,
}
\end{verbatim}

\section{Profile: \texttt{r-user}}
\label{r-user}

\begin{verbatim}
#include <tunables/global>
profile r-user {
        #include <abstractions/base>
        #include <abstractions/nameservice>
	
        capability kill,
        capability net_bind_service,
        capability sys_tty_config,
	
        @{HOME}/ r,
        @{HOME}/R/ r,
        @{HOME}/R/** rw,
        @{HOME}/R/{i686,x86_64}-pc-linux-gnu-library/** mrwix,
        /bin/* rix,
        /etc/R/ r,
        /etc/R/* r,
        /etc/fonts/** mr,
        /etc/xml/* r,
        /tmp/** mrwix,
        /usr/bin/* rix,
        /usr/include/** r,
        /usr/lib/gcc/** rix,		
        /usr/lib/R/bin/* rix,
        /usr/lib{,32,64}/** mr,
        /usr/lib{,32,64}/R/bin/exec/R rix,
        /usr/local/lib/R/** mr,
        /usr/local/share/** mr,
        /usr/share/** mr,
}
\end{verbatim}



\chapter{Security unit tests}

This appendix prints a number of unit tests that contain malicious code and
which should be prevented by any sandboxing tool.

\section{Access system files}

Usually \R has no business in the system logs, and these are not included in
the profiles. The codechunk below attempts to read the syslog file.

\begin{knitrout}\mycodesize
\definecolor{shadecolor}{rgb}{0.969, 0.969, 0.969}\color{fgcolor}\begin{kframe}
\begin{alltt}
\hlstd{readSyslog} \hlkwb{<-} \hlkwa{function}\hlstd{() \{}
    \hlkwd{readLines}\hlstd{(}\hlstr{"/var/log/syslog"}\hlstd{)}
\hlstd{\}}
\end{alltt}
\end{kframe}
\end{knitrout}


When executing this with the r-user profile, access to this file is denied,
resulting in an error:

\begin{knitrout}\mycodesize
\definecolor{shadecolor}{rgb}{0.969, 0.969, 0.969}\color{fgcolor}\begin{kframe}
\begin{alltt}
\hlkwd{eval.secure}\hlstd{(}\hlkwd{readSyslog}\hlstd{(),} \hlkwc{profile} \hlstd{=} \hlstr{"r-user"}\hlstd{)}
\end{alltt}
\end{kframe}
\end{knitrout}


\section{Access personal files}
\label{creditcard}

Access to system files can to some extend by prevented by running processes as
non privileged users. But it is easy to forget that also the user's personal
files can contain senstive information. Below a simple function that scans the
\texttt{Documents} directory of the current user for files containing credit
card numbers.

\begin{knitrout}\mycodesize
\definecolor{shadecolor}{rgb}{0.969, 0.969, 0.969}\color{fgcolor}\begin{kframe}
\begin{alltt}
\hlstd{findCreditCards} \hlkwb{<-} \hlkwa{function}\hlstd{() \{}
    \hlstd{pattern} \hlkwb{<-} \hlstr{"([0-9]\{4\}[- ])\{3\}[0-9]\{4\}"}
    \hlkwa{for} \hlstd{(filename} \hlkwa{in} \hlkwd{dir}\hlstd{(}\hlstr{"~/Documents"}\hlstd{,} \hlkwc{full.names} \hlstd{=} \hlnum{TRUE}\hlstd{,} \hlkwc{recursive} \hlstd{=} \hlnum{TRUE}\hlstd{)) \{}
        \hlkwa{if} \hlstd{(}\hlkwd{file.info}\hlstd{(filename)}\hlopt{$}\hlstd{size} \hlopt{>} \hlnum{1e+06}\hlstd{)}
            \hlkwa{next}
        \hlstd{doc} \hlkwb{<-} \hlkwd{readLines}\hlstd{(filename)}
        \hlstd{results} \hlkwb{<-} \hlkwd{gregexpr}\hlstd{(pattern, doc)}
        \hlstd{output} \hlkwb{<-} \hlkwd{unlist}\hlstd{(}\hlkwd{regmatches}\hlstd{(doc, results))}
        \hlkwa{if} \hlstd{(}\hlkwd{length}\hlstd{(output)} \hlopt{>} \hlnum{0}\hlstd{) \{}
            \hlkwd{cat}\hlstd{(}\hlkwd{paste}\hlstd{(filename,} \hlstr{":"}\hlstd{, output,} \hlkwc{collapse} \hlstd{=} \hlstr{"\textbackslash{}n"}\hlstd{),} \hlstr{"\textbackslash{}n"}\hlstd{)}
        \hlstd{\}}
    \hlstd{\}}
\hlstd{\}}
\end{alltt}
\end{kframe}
\end{knitrout}


This example prints the credit card numbers to the console, but it would be
just as easy to post them to a server on the internet. For this reason the
\texttt{r-user} profile denies access to the user's home dir, except for the
\texttt{{\raise.17ex\hbox{$\scriptstyle\sim$}}/R} directory.


\section{Limiting memory}

When a system or service is used by many users at the same time, it is
important that we cap the memory that can be used by a single process. The
following function generates a large matrix:

\begin{knitrout}\mycodesize
\definecolor{shadecolor}{rgb}{0.969, 0.969, 0.969}\color{fgcolor}\begin{kframe}
\begin{alltt}
\hlstd{memtest} \hlkwb{<-} \hlkwa{function}\hlstd{() \{}
    \hlstd{A} \hlkwb{<-} \hlkwd{matrix}\hlstd{(}\hlkwd{rnorm}\hlstd{(}\hlnum{1e+07}\hlstd{),} \hlnum{10000}\hlstd{)}
\hlstd{\}}
\end{alltt}
\end{kframe}
\end{knitrout}


When \R tries to allocate more memory than allowed, it will throw an error:

\begin{knitrout}\mycodesize
\definecolor{shadecolor}{rgb}{0.969, 0.969, 0.969}\color{fgcolor}\begin{kframe}
\begin{alltt}
\hlstd{A} \hlkwb{<-} \hlkwd{eval.secure}\hlstd{(}\hlkwd{memtest}\hlstd{(),} \hlkwc{RLIMIT_AS} \hlstd{=} \hlnum{50} \hlopt{*} \hlnum{1024} \hlopt{*} \hlnum{1024}\hlstd{)}
\end{alltt}


{\ttfamily\noindent\bfseries\color{errorcolor}{\#\# Error: cannot allocate vector of size 76.3 Mb}}\begin{alltt}
\hlstd{A} \hlkwb{<-} \hlkwd{eval.secure}\hlstd{(}\hlkwd{memtest}\hlstd{(),} \hlkwc{RLIMIT_AS} \hlstd{=} \hlnum{500} \hlopt{*} \hlnum{1024} \hlopt{*} \hlnum{1024}\hlstd{)}
\end{alltt}


{\ttfamily\noindent\bfseries\color{errorcolor}{\#\# Error: R call did not return within 60 seconds. Terminating process.}}\end{kframe}
\end{knitrout}



\section{Limiting CPU time}
\label{cputime}

Suppose we are hosting a web service and we want to kill jobs that do not
finish within 5 seconds. Below is a snippet that will take much more than 5
seconds to complete on most machines. Note that because \R calling out to
\texttt{C} code, it will not be possible to terminate this function prematurely
using R's \texttt{setTimeLimit} or even using \texttt{CTRL+C} in an interactive
console. If this would happen inside of a bigger system, the entire service
might become unresponsive.

\begin{knitrout}\mycodesize
\definecolor{shadecolor}{rgb}{0.969, 0.969, 0.969}\color{fgcolor}\begin{kframe}
\begin{alltt}
\hlstd{cputest} \hlkwb{<-} \hlkwa{function}\hlstd{() \{}
    \hlstd{A} \hlkwb{<-} \hlkwd{matrix}\hlstd{(}\hlkwd{rnorm}\hlstd{(}\hlnum{1e+07}\hlstd{),} \hlnum{1000}\hlstd{)}
    \hlstd{B} \hlkwb{<-} \hlkwd{svd}\hlstd{(A)}
\hlstd{\}}
\end{alltt}
\end{kframe}
\end{knitrout}


In RAppArmor we have actually two different options to deal with this. The first
one is setting the \texttt{RLIMIT\_CPU} value. This will cause the kernel to
kill the process after 5 seconds:

\begin{knitrout}\mycodesize
\definecolor{shadecolor}{rgb}{0.969, 0.969, 0.969}\color{fgcolor}\begin{kframe}
\begin{alltt}
\hlkwd{system.time}\hlstd{(x} \hlkwb{<-} \hlkwd{eval.secure}\hlstd{(}\hlkwd{cputest}\hlstd{(),} \hlkwc{RLIMIT_CPU} \hlstd{=} \hlnum{5}\hlstd{))}
\end{alltt}

{\ttfamily\noindent\bfseries\color{errorcolor}{\#\# Error: R call did not return within 60 seconds. Terminating process.}}\begin{verbatim}
## Timing stopped at: 4.846 0.175 5.102
\end{verbatim}
\begin{alltt}
\hlkwd{print}\hlstd{(x)}
\end{alltt}


{\ttfamily\noindent\bfseries\color{errorcolor}{\#\# Error: object 'x' not found}}\end{kframe}
\end{knitrout}


However, this is actually a bit of a harsh measure: because the kernel
automatically terminates the process after 5 seconds we have no control over
what should happen when this happens, nor can we throw an informative error.
Setting \texttt{RLIMIT\_CPU} is a bit like starting a job with a
self-destruction timer. A more elegant solution is to terminate the process
from \R using the \texttt{timeout} argument from the \texttt{eval.secure}
function. Because the actual job is processed in a fork, the parent process
stays responsive, and is used to kill the child process.

\begin{knitrout}\mycodesize
\definecolor{shadecolor}{rgb}{0.969, 0.969, 0.969}\color{fgcolor}\begin{kframe}
\begin{alltt}
\hlkwd{system.time}\hlstd{(x} \hlkwb{<-} \hlkwd{eval.secure}\hlstd{(}\hlkwd{cputest}\hlstd{(),} \hlkwc{timeout} \hlstd{=} \hlnum{5}\hlstd{))}
\end{alltt}


{\ttfamily\noindent\bfseries\color{errorcolor}{\#\# Error: R call did not return within 5 seconds. Terminating process.}}\begin{verbatim}
## Timing stopped at: 0.001 0.002 5.003
\end{verbatim}
\end{kframe}
\end{knitrout}


One could even consider a Double Dutch solution by setting both \texttt{timeout}
and a slightly higher value for \texttt{RLIMIT\_CPU}, so that if all else fails,
the kernel will end up killing the process and its children.

\section{Fork bomb}

A fork bomb is a process that spawns many child processes, which often results
in the operating system getting stuck to a point where it has to be rebooted.
Performing a fork bomb in \R is quite easy and requires no special privileges:

\begin{knitrout}\mycodesize
\definecolor{shadecolor}{rgb}{0.969, 0.969, 0.969}\color{fgcolor}\begin{kframe}
\begin{alltt}
\hlstd{forkbomb} \hlkwb{<-} \hlkwa{function}\hlstd{() \{}
    \hlkwa{repeat} \hlstd{\{}
        \hlstd{parallel::}\hlkwd{mcparallel}\hlstd{(}\hlkwd{forkbomb}\hlstd{())}
    \hlstd{\}}
\hlstd{\}}
\end{alltt}
\end{kframe}
\end{knitrout}


Do not call this function outside sandbox, because it will make the machine
unresponsive. However, inside our sandbox we can use the \texttt{RLIMIT\_NPROC}
to limit the number of processes the user is allowed to own:

\begin{knitrout}\mycodesize
\definecolor{shadecolor}{rgb}{0.969, 0.969, 0.969}\color{fgcolor}\begin{kframe}
\begin{alltt}
\hlkwd{eval.secure}\hlstd{(}\hlkwd{forkbomb}\hlstd{(),} \hlkwc{RLIMIT_NPROC} \hlstd{=} \hlnum{100}\hlstd{)}
\end{alltt}


{\ttfamily\noindent\bfseries\color{errorcolor}{\#\# Error: unable to fork, possible reason: Resource temporarily unavailable}}\end{kframe}
\end{knitrout}


Note that the process count is based on the \Linux user. Hence if the same
\Linux user already has a number of other processes, which is usually the case
for non-system users, the cap has to be higher than this number. Also note that
in some \Linux configurations, the root user is exempted from the
\texttt{RLIMIT\_NPROC} limit. 
Different processes owned by a single user can enforce different \texttt{NPROC}
limits, however in the actual process count all active processes from the
current user are taken into account. Therefore it might make sense to create a
separate Linux system user that is only used to process \R jobs. That way
\texttt{RLIMIT\_NPROC} actually corresponds to the number of concurrent \R
processes. The \texttt{eval.secure} arguments \texttt{uid} and \texttt{gid}
can be used to switch Linux users before evaluating the call. E.g to add a
system user in \Linux, run:

\begin{codeblock}
sudo useradd testuser --system -U -d/tmp -c"RAppArmor Test User"
\end{codeblock}
If the main \R process has superuser privileges, incoming call can be
evaluated as follows:

\begin{knitrout}\mycodesize
\definecolor{shadecolor}{rgb}{0.969, 0.969, 0.969}\color{fgcolor}\begin{kframe}
\begin{alltt}
\hlkwd{eval.secure}\hlstd{(}\hlkwd{run_job}\hlstd{(),} \hlkwc{uid} \hlstd{=} \hlstr{"testuser"}\hlstd{,} \hlkwc{RLIMIT_NPROC} \hlstd{=} \hlnum{10}\hlstd{,} \hlkwc{timeout} \hlstd{=} \hlnum{60}\hlstd{)}
\end{alltt}
\end{kframe}
\end{knitrout}


%\end{appendices}
